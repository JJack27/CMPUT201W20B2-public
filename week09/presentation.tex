% Created 2020-03-10 Tue 11:30
% Intended LaTeX compiler: pdflatex
\documentclass[11pt]{article}
\usepackage[utf8]{inputenc}
\usepackage[T1]{fontenc}
\usepackage{graphicx}
\usepackage{grffile}
\usepackage{longtable}
\usepackage{wrapfig}
\usepackage{rotating}
\usepackage[normalem]{ulem}
\usepackage{amsmath}
\usepackage{textcomp}
\usepackage{amssymb}
\usepackage{capt-of}
\usepackage{hyperref}
\author{Abram Hindle}
\date{\today}
\title{CMPUT201W20B2 Week 9}
\hypersetup{
 pdfauthor={Abram Hindle},
 pdftitle={CMPUT201W20B2 Week 9},
 pdfkeywords={},
 pdfsubject={},
 pdfcreator={Emacs 25.2.2 (Org mode 9.1.6)}, 
 pdflang={English}}
\begin{document}

\maketitle
\tableofcontents


\section{Week9}
\label{sec:orgc4a2973}
\subsection{Copyright Statement}
\label{sec:org30925d6}

If you are in CMPUT201 at UAlberta this code is released in the public
domain to you.

Otherwise it is (c) 2020 Abram Hindle, Hazel Campbell AGPL3.0+

\subsubsection{License}
\label{sec:org28647cd}

Week 3 notes
Copyright (C) 2020 Abram Hindle, Hazel Campbell

This program is free software: you can redistribute it and/or modify
it under the terms of the GNU Affero General Public License as
published by the Free Software Foundation, either version 3 of the
License, or (at your option) any later version.

This program is distributed in the hope that it will be useful,
but WITHOUT ANY WARRANTY; without even the implied warranty of
MERCHANTABILITY or FITNESS FOR A PARTICULAR PURPOSE.  See the
GNU Affero General Public License for more details.

You should have received a copy of the GNU Affero General Public License
along with this program.  If not, see \url{https://www.gnu.org/licenses/}.


\subsubsection{Hazel Code is licensed under AGPL3.0+}
\label{sec:org3d0fff1}

Hazel's code is also found here
\url{https://github.com/hazelybell/examples/tree/C-2020-01}

Hazel code is licensed: The example code is licensed under the AGPL3+
license, unless otherwise noted.

\subsection{Init ORG-MODE}
\label{sec:orgef3028d}

\begin{verbatim}
;; I need this for org-mode to work well
;; If we have a new org-mode use ob-shell
;; otherwise use ob-sh --- but not both!
(if (require 'ob-shell nil 'noerror)
  (progn
    (org-babel-do-load-languages 'org-babel-load-languages '((shell . t))))
  (progn
    (require 'ob-sh)
    (org-babel-do-load-languages 'org-babel-load-languages '((sh . t)))))
(org-babel-do-load-languages 'org-babel-load-languages '((C . t)))
(org-babel-do-load-languages 'org-babel-load-languages '((python . t)))
(setq org-src-fontify-natively t)
(setq org-confirm-babel-evaluate nil) ;; danger!
(custom-set-faces
 '(org-block ((t (:inherit shadow :foreground "black"))))
 '(org-code ((t (:inherit shadow :foreground "black")))))
\end{verbatim}

\subsubsection{Org export}
\label{sec:org2da4716}
\begin{verbatim}
(org-html-export-to-html)
(org-latex-export-to-pdf)
(org-ascii-export-to-ascii)
\end{verbatim}


\subsubsection{Org Template}
\label{sec:orge1fe7d2}
Copy and paste this to demo C

\begin{verbatim}
#include <stdio.h>

int main(int argc, char**argv) {
    return 0;
}
\end{verbatim}

\subsection{Remember how to compile?}
\label{sec:orgccfc772}

gcc  -std=c99 -pedantic -Wall -Wextra -ftrapv -ggdb3 -o programname programname.c


\subsection{How C uses memory}
\label{sec:orgdab645e}

\url{https://docs.google.com/document/d/1wcBnHZEaW4ukcZNlMNZJdbxkyPOQFGQhtANpqW5tx\_8/edit?usp=sharing}

\subsubsection{C's memory types}
\label{sec:org908b849}

\begin{itemize}
\item File memory (code or text) 
\begin{itemize}
\item handled by the compiler, the linker, the loader, and the OS
\item contains code, string literals, constants
\item read only
\item big
\end{itemize}
\item static 
\begin{itemize}
\item your globals and static vars
\item static or auto
\item big
\end{itemize}
\item Call stack or stack
\begin{itemize}
\item fast local memory for a function
\item auto
\item small amount
\end{itemize}
\item Heap
\begin{itemize}
\item general storage / memory for everything
\item auto / static/ whatever
\item huge
\end{itemize}
\item registers
\begin{itemize}
\item not actually CPU registers -- but you hope!
\item meant to be moved around for optimization reasons
\item compiler dictated
\item 'register' variables -- don't make pointers to these
\item meant to be fast, might be in stack might be in actual registers
\begin{itemize}
\item no guarantees
\item your compiler w/ optimizer is typically better.
\end{itemize}
\end{itemize}
\end{itemize}
\subsubsection{Type qualifiers}
\label{sec:org36426c9}

Add these to types to communicate to other programmers and the
compiler.

\begin{itemize}
\item const - don't change it and I promise not to change it (but it might
change because I messed up with malloc)
\item volatile - this value could change so if you need it read it
immediately and use it. It could even change while you use it. Don't
rely on it staying the same. Something external could be changing it.
\item restrict - I promise that this pointer is the only pointer to the
thing it is pointing to. Speed hacks that rarely work.
\end{itemize}

\subsubsection{Extern}
\label{sec:orgefb5f53}

Extern says that we have an external implementation or allocation for
a variable or function. But if you define the body then you've gone
and done it. It is fine. Extern means you plan to share your
implementation with other source files.

\subsection{Preprocessor stuff like if-def}
\label{sec:org0ca5430}
The preprocessor deals with all the lines that you start with an
octalthrope or hash mark: \#

The preprocessor lets you define symbols, macros, and include
files.

\subsubsection{\#ifdef}
\label{sec:orgb69ab94}

\begin{verbatim}
/* #ifdef   IF DEFined
 * 
 * If whatever comes after the #ifdef IS
 * defined, then all the code until the #endif
 * be treated normally.
 * 
 * If whatever comes afer the #ifdef is NOT
 * defined, then all the code until the #endif
 * will be SKIPPED, and not compiled at all.
 * 
 * #else    
 * Similar to else but in the preprocessor
 */

#include <stdio.h>
#include <stdio.h>

#ifdef ENABLE_NONSENSE
this is not even real c code!

all of this gets skipped
#endif

// Try commenting/uncommenting the following:
#define TURBO
#define DEBUG



int main() {
    
#ifdef TURBO
    int value = 27;
#else
    int value = 32;
#endif /* def TURBO */

    printf("Value is %d!\n", value);
    
#ifdef TURBO
    printf("Turbo is on!\n");
#endif /* def TURBO */
    
#ifdef DEBUG
    printf("Reached end of main, quitting!\n");
#endif
    
    return 0;
}
\end{verbatim}

\begin{verbatim}
Value is 27!
Turbo is on!
Reached end of main, quitting!
\end{verbatim}



\subsubsection{\#ifndef}
\label{sec:org1008e6a}

\begin{verbatim}
/* #ifndef   IF Not DEFined
 * 
 * If whatever comes after the #ifndef is NOT
 * defined, then all the code until the #endif
 * be treated normally.
 * 
 * If whatever comes afer the #ifndef IS
 * defined, then all the code until the #endif
 * will be SKIPPED, and not compiled at all.
 */

#include <stdio.h>

#ifdef THING
order matters here
#ifdef OTHER_THING
this will only appear in the compiled program if both things are defined

order doesnt matter here
#endif
order matters here too
#endif

#ifndef MAIN_DEFINED
#define MAIN_DEFINED
int main() {
    printf("Main 1!\n");
    return 0;
}
#endif

#ifndef MAIN_DEFINED
#define MAIN_DEFINED
int main() {
    printf("Main 2!\n");
    return 0;
}
#endif
\end{verbatim}

\begin{verbatim}
Main 1!
\end{verbatim}

\subsubsection{Guards}
\label{sec:org1fe58b6}

\begin{verbatim}
/* Guards:
 * 
 * The purpose of the guard is to ensure that
 * IF the header is included more than once,
 * everything in it will be SKIPPED the second,
 * third, fourth, etc. time the header is
 * included.
 * 
 * For example, we might have main.c which
 * includes io.h which includes data.h,
 * as well as incuding data.h directly.
 * 
 * In such a situation, data.h gets included
 * TWICE in main.c, which would produce errors
 * without guards!
 */

/* #ifndef   IF Not DEFined
 * 
 * If whatever comes after the #ifndef is NOT
 * defined, then all the code until the #endif
 * be treated normally.
 * 
 * If whatever comes afer the #ifndef IS
 * defined, then all the code until the #endif
 * will be SKIPPED, and not compiled at all.
 */

#ifndef _GUARDS_H_

#define _GUARDS_H_


// # ends if
#endif /* ndef _GUARDS_H_ */
\end{verbatim}


\begin{enumerate}
\item No Guards
\label{sec:org7cd7f39}

What if we don't have a guard?

We could redefine functions. Make conflicting types. Get in infinite
include loops.

\begin{verbatim}
#include <stdio.h>
/* No Guards:
 * 
 */
#define GUARDS "cool"

#define GUARDS "awesome"

int main() {
    puts(GUARDS);
}
\end{verbatim}

\begin{verbatim}
awesome
\end{verbatim}

\item W/ Guards
\label{sec:orgbea1419}

What if we have a guard?

We only define once :)

\begin{verbatim}
#include <stdio.h>
/* No Guards:
 * 
 */
#ifndef GUARDS
#define GUARDS "cool"
#endif

#ifndef GUARDS
#define GUARDS "awesome"
#endif

int main() {
    puts(GUARDS);
}
\end{verbatim}

\begin{verbatim}
cool
\end{verbatim}
\end{enumerate}

\subsubsection{Multiple Files?}
\label{sec:org67d71ae}

How does stdio.h work?

\url{file:///usr/include/stdio.h}

It defines definitions, macros, and prototypes for the stdio library.
The linker will link your executable to that library that was already
compiled.

.h files help us organize C programs by including definitions for the
object files and libraries that we will create.

Libc or glibc contains the implemention of those definitions.
libc.so.6 => /lib/x86\(_{\text{64}}\)-linux-gnu/libc.so.6 (0x00007f919f994000)

libc is composed of many .c files compiled into .o object files and
then combined into a library. A library is like an executable that
other executables rely on for code. malloc is defined in malloc.c and
has a malloc.h file!

Typically if I make a library I will make a .h file so the definitions
can be shared with other .c files. But the implementation of the functions
will go into a .c file that includes that .h as well.

\begin{itemize}
\item main.c
\begin{itemize}
\item \#include "library.h"
\item relies on library.o
\end{itemize}
\item library.c
\begin{itemize}
\item \#include "library.h"
\item makes library.o
\end{itemize}
\item library.h
\begin{itemize}
\item defines functions and definitions from library.c
\end{itemize}
\end{itemize}

\subsubsection{Example}
\label{sec:org054ba95}

This is a useful function to check if scanf read 1 or more elements
and didn't read EOF.

\url{./checkinput.c}

\begin{verbatim}
#include "checkinput.h"
#include <stdio.h>
#include <stdlib.h>
/* checkInput: given the result of scanf check if it 
 * 0 elements read or EOF. If so exit(1) with a warning.
 *
 */
void checkInput(int err) {
  if (!err || err == EOF) {
    printf("\nInvalid input!\n");
    exit(1);
  }
}
\end{verbatim}

\url{./checkinput.h}

\begin{verbatim}
// Have a guard to ensure that we don't include it multiple times.
#ifndef _CHECKINPUT_H_
/* checkInput: given the result of scanf check if it 
 * 0 elements read or EOF. If so exit(1) with a warning.
 *
 */
#define _CHECKINPUT_H_
void checkInput(int err); // a prototype!
#endif
\end{verbatim}

\url{./checkinput-driver.c}

\begin{verbatim}
#include "checkinput.h"
#include <stdio.h>
#include "checkinput.h"

int main() {
  int input;
  checkInput(scanf("%d", &input));  
  puts("Good Input!");
}
\end{verbatim}

\begin{enumerate}
\item Compiling Multiple Files
\label{sec:org2f354ab}

OK now we compile it. The main is the last to compile and it needs all the .o files.

All the .c files that don't contain main need to be compiled to object
files. Use the -c flags to do this.

\begin{verbatim}
# build checkinput.o
gcc  -std=c99 -pedantic -Wall -Wextra -ftrapv -ggdb3 \
       -c checkinput.c
# build checkinput-driver and link it to checkinput.o
gcc  -std=c99 -pedantic -Wall -Wextra -ftrapv -ggdb3 \
       -o checkinput-driver checkinput-driver.c \
       checkinput.o
\end{verbatim}

Test drive it

\begin{verbatim}
echo   | ./checkinput-driver
echo X | ./checkinput-driver
echo 1 | ./checkinput-driver
echo 1 | ./checkinput-driver
\end{verbatim}

\begin{verbatim}

Invalid input!

Invalid input!
Good Input!
Good Input!
\end{verbatim}


\begin{verbatim}
ldd ./checkinput-driver
\end{verbatim}

\begin{verbatim}
linux-vdso.so.1 (0x00007ffe85be0000)
libc.so.6 => /lib/x86_64-linux-gnu/libc.so.6 (0x00007f919f994000)
/lib64/ld-linux-x86-64.so.2 (0x00007f919ff87000)
\end{verbatim}
\end{enumerate}

\subsubsection{Example Datastructure}
\label{sec:org566a4ef}

This is a useful function to check if scanf read 1 or more elements
and didn't read EOF.

\url{./coolbears.c}

\begin{verbatim}
#define _POSIX_C_SOURCE 200809L // <-- needed for strdup
#include "coolbears.h"
#include <stdio.h>
#include <stdlib.h>
#include <string.h>
// hiding struct details from other programmers
// I DONT TRUST THEM. Especially Hazel ;-) (don't tell hazel)
struct coolbear_t {
    char * name;
    float temperature;
};

CoolBear createCoolBear(char * name, float temperature) {
    CoolBear coolbear = malloc(sizeof(*coolbear));
    coolbear->name = strdup(name);
    coolbear->temperature = temperature;
    return coolbear;
}
void freeCoolBear(CoolBear coolBear) {
    if (coolBear == NULL) {
        abort();
    }
    if (coolBear->name != NULL) {
        free(coolBear->name);
    }
    free(coolBear);
}
char * getNameCoolBear(CoolBear coolbear) {
    return coolbear->name;
}
float    getTemperatureCoolBear(CoolBear coolbear) {
   return coolbear->temperature;
}
// NO MAIN!
\end{verbatim}



\url{./coolbears.h}

\begin{verbatim}
// Have a guard to ensure that we don't include it multiple times.
#ifndef _COOLBEARS_H_
/* checkInput: given the result of scanf check if it 
 * 0 elements read or EOF. If so exit(1) with a warning.
 *
 */
#define _COOLBEARS_H_
struct coolbear_t; // Forward declaration -- I am not sharing details!
typedef struct coolbear_t * CoolBear; // Struct point as type

CoolBear createCoolBear(char * name, float temperature); // a prototype!
void     freeCoolBear(CoolBear coolBear); // a prototype!
char *   getNameCoolBear(CoolBear coolbear); // a prototype!
float    getTemperatureCoolBear(CoolBear coolbear); // a prototype!

#endif
\end{verbatim}

\url{./coolbears-driver.c}

\begin{verbatim}
#include "coolbears.h"
#include <stdio.h>


int main() {
  CoolBear ziggy = createCoolBear("Ziggy",-23.0 /* C */);
  CoolBear kevin = createCoolBear("Kevin",-32.0 /* C */);
  CoolBear coolest = (getTemperatureCoolBear(ziggy) < 
                      getTemperatureCoolBear(kevin))? ziggy : kevin;
  printf("The coolest bear is %s\n", getNameCoolBear( coolest ));
  // // we actually don't know about name so we can't reference it below
  // printf("The coolest bear is %s\n", getNameCoolBear( coolest->name ));
  freeCoolBear(ziggy);
  freeCoolBear(kevin);
}
\end{verbatim}

Compile it. -c the coolbears.c to make coolbears.o and then 
compile coolbears-driver.c

coolbears-driver.c has no clue how to access 

\begin{verbatim}
# build coolbears.o
gcc  -std=c99 -pedantic -Wall -Wextra -ftrapv -ggdb3 \
       -c coolbears.c
# build coolbears-driver and link it to coolbears.o
gcc  -std=c99 -pedantic -Wall -Wextra -ftrapv -ggdb3 \
       -o coolbears-driver coolbears-driver.c \
       coolbears.o 
./coolbears-driver
\end{verbatim}

\begin{verbatim}
The coolest bear is Kevin
\end{verbatim}

If we access coolest->name we get:

\begin{verbatim}
coolbears-driver.c: In function ‘main’:
coolbears-driver.c:11:62: error: dereferencing pointer to incomplete type ‘struct coolbear_t’
   printf("The coolest bear is %s\n", getNameCoolBear( coolest->name ));
\end{verbatim}


\subsubsection{What is the preprocessor doing?}
\label{sec:org03a0ca8}

Let's use the -E flag to see what checkinput.c becomes

This output contains glibc headers for stdio.h and stdlib.h these
should be under the GPLV3 (c) the Glibc project and GNU project.

If you want more preprocessor options checkout:

\url{https://gcc.gnu.org/onlinedocs/gcc-5.2.0/gcc/Preprocessor-Options.html}

\begin{verbatim}
# build checkinput.o
gcc -E -std=c99 -pedantic -Wall -Wextra -ftrapv -ggdb3 \
       checkinput.c
\end{verbatim}

\begin{verbatim}
# 1 "checkinput.c"
# 1 "/home/hindle1/projects/CMPUT201W20/2020-01/CMPUT201W20B2-public/week09//"
# 1 "<built-in>"
#define __STDC__ 1
#define __STDC_VERSION__ 199901L
#define __STDC_HOSTED__ 1
#define __GNUC__ 7
#define __GNUC_MINOR__ 4
#define __GNUC_PATCHLEVEL__ 0
#define __VERSION__ "7.4.0"
#define __ATOMIC_RELAXED 0
#define __ATOMIC_SEQ_CST 5
#define __ATOMIC_ACQUIRE 2
#define __ATOMIC_RELEASE 3
#define __ATOMIC_ACQ_REL 4
#define __ATOMIC_CONSUME 1
#define __pic__ 2
#define __PIC__ 2
#define __pie__ 2
#define __PIE__ 2
#define __FINITE_MATH_ONLY__ 0
#define _LP64 1
#define __LP64__ 1
#define __SIZEOF_INT__ 4
#define __SIZEOF_LONG__ 8
#define __SIZEOF_LONG_LONG__ 8
#define __SIZEOF_SHORT__ 2
#define __SIZEOF_FLOAT__ 4
#define __SIZEOF_DOUBLE__ 8
#define __SIZEOF_LONG_DOUBLE__ 16
#define __SIZEOF_SIZE_T__ 8
#define __CHAR_BIT__ 8
#define __BIGGEST_ALIGNMENT__ 16
#define __ORDER_LITTLE_ENDIAN__ 1234
#define __ORDER_BIG_ENDIAN__ 4321
#define __ORDER_PDP_ENDIAN__ 3412
#define __BYTE_ORDER__ __ORDER_LITTLE_ENDIAN__
#define __FLOAT_WORD_ORDER__ __ORDER_LITTLE_ENDIAN__
#define __SIZEOF_POINTER__ 8
#define __SIZE_TYPE__ long unsigned int
#define __PTRDIFF_TYPE__ long int
#define __WCHAR_TYPE__ int
#define __WINT_TYPE__ unsigned int
#define __INTMAX_TYPE__ long int
#define __UINTMAX_TYPE__ long unsigned int
#define __CHAR16_TYPE__ short unsigned int
#define __CHAR32_TYPE__ unsigned int
#define __SIG_ATOMIC_TYPE__ int
#define __INT8_TYPE__ signed char
#define __INT16_TYPE__ short int
#define __INT32_TYPE__ int
#define __INT64_TYPE__ long int
#define __UINT8_TYPE__ unsigned char
#define __UINT16_TYPE__ short unsigned int
#define __UINT32_TYPE__ unsigned int
#define __UINT64_TYPE__ long unsigned int
#define __INT_LEAST8_TYPE__ signed char
#define __INT_LEAST16_TYPE__ short int
#define __INT_LEAST32_TYPE__ int
#define __INT_LEAST64_TYPE__ long int
#define __UINT_LEAST8_TYPE__ unsigned char
#define __UINT_LEAST16_TYPE__ short unsigned int
#define __UINT_LEAST32_TYPE__ unsigned int
#define __UINT_LEAST64_TYPE__ long unsigned int
#define __INT_FAST8_TYPE__ signed char
#define __INT_FAST16_TYPE__ long int
#define __INT_FAST32_TYPE__ long int
#define __INT_FAST64_TYPE__ long int
#define __UINT_FAST8_TYPE__ unsigned char
#define __UINT_FAST16_TYPE__ long unsigned int
#define __UINT_FAST32_TYPE__ long unsigned int
#define __UINT_FAST64_TYPE__ long unsigned int
#define __INTPTR_TYPE__ long int
#define __UINTPTR_TYPE__ long unsigned int
#define __has_include(STR) __has_include__(STR)
#define __has_include_next(STR) __has_include_next__(STR)
#define __GXX_ABI_VERSION 1011
#define __SCHAR_MAX__ 0x7f
#define __SHRT_MAX__ 0x7fff
#define __INT_MAX__ 0x7fffffff
#define __LONG_MAX__ 0x7fffffffffffffffL
#define __LONG_LONG_MAX__ 0x7fffffffffffffffLL
#define __WCHAR_MAX__ 0x7fffffff
#define __WCHAR_MIN__ (-__WCHAR_MAX__ - 1)
#define __WINT_MAX__ 0xffffffffU
#define __WINT_MIN__ 0U
#define __PTRDIFF_MAX__ 0x7fffffffffffffffL
#define __SIZE_MAX__ 0xffffffffffffffffUL
#define __SCHAR_WIDTH__ 8
#define __SHRT_WIDTH__ 16
#define __INT_WIDTH__ 32
#define __LONG_WIDTH__ 64
#define __LONG_LONG_WIDTH__ 64
#define __WCHAR_WIDTH__ 32
#define __WINT_WIDTH__ 32
#define __PTRDIFF_WIDTH__ 64
#define __SIZE_WIDTH__ 64
#define __INTMAX_MAX__ 0x7fffffffffffffffL
#define __INTMAX_C(c) c ## L
#define __UINTMAX_MAX__ 0xffffffffffffffffUL
#define __UINTMAX_C(c) c ## UL
#define __INTMAX_WIDTH__ 64
#define __SIG_ATOMIC_MAX__ 0x7fffffff
#define __SIG_ATOMIC_MIN__ (-__SIG_ATOMIC_MAX__ - 1)
#define __SIG_ATOMIC_WIDTH__ 32
#define __INT8_MAX__ 0x7f
#define __INT16_MAX__ 0x7fff
#define __INT32_MAX__ 0x7fffffff
#define __INT64_MAX__ 0x7fffffffffffffffL
#define __UINT8_MAX__ 0xff
#define __UINT16_MAX__ 0xffff
#define __UINT32_MAX__ 0xffffffffU
#define __UINT64_MAX__ 0xffffffffffffffffUL
#define __INT_LEAST8_MAX__ 0x7f
#define __INT8_C(c) c
#define __INT_LEAST8_WIDTH__ 8
#define __INT_LEAST16_MAX__ 0x7fff
#define __INT16_C(c) c
#define __INT_LEAST16_WIDTH__ 16
#define __INT_LEAST32_MAX__ 0x7fffffff
#define __INT32_C(c) c
#define __INT_LEAST32_WIDTH__ 32
#define __INT_LEAST64_MAX__ 0x7fffffffffffffffL
#define __INT64_C(c) c ## L
#define __INT_LEAST64_WIDTH__ 64
#define __UINT_LEAST8_MAX__ 0xff
#define __UINT8_C(c) c
#define __UINT_LEAST16_MAX__ 0xffff
#define __UINT16_C(c) c
#define __UINT_LEAST32_MAX__ 0xffffffffU
#define __UINT32_C(c) c ## U
#define __UINT_LEAST64_MAX__ 0xffffffffffffffffUL
#define __UINT64_C(c) c ## UL
#define __INT_FAST8_MAX__ 0x7f
#define __INT_FAST8_WIDTH__ 8
#define __INT_FAST16_MAX__ 0x7fffffffffffffffL
#define __INT_FAST16_WIDTH__ 64
#define __INT_FAST32_MAX__ 0x7fffffffffffffffL
#define __INT_FAST32_WIDTH__ 64
#define __INT_FAST64_MAX__ 0x7fffffffffffffffL
#define __INT_FAST64_WIDTH__ 64
#define __UINT_FAST8_MAX__ 0xff
#define __UINT_FAST16_MAX__ 0xffffffffffffffffUL
#define __UINT_FAST32_MAX__ 0xffffffffffffffffUL
#define __UINT_FAST64_MAX__ 0xffffffffffffffffUL
#define __INTPTR_MAX__ 0x7fffffffffffffffL
#define __INTPTR_WIDTH__ 64
#define __UINTPTR_MAX__ 0xffffffffffffffffUL
#define __GCC_IEC_559 2
#define __GCC_IEC_559_COMPLEX 2
#define __FLT_EVAL_METHOD__ 0
#define __FLT_EVAL_METHOD_TS_18661_3__ 0
#define __DEC_EVAL_METHOD__ 2
#define __FLT_RADIX__ 2
#define __FLT_MANT_DIG__ 24
#define __FLT_DIG__ 6
#define __FLT_MIN_EXP__ (-125)
#define __FLT_MIN_10_EXP__ (-37)
#define __FLT_MAX_EXP__ 128
#define __FLT_MAX_10_EXP__ 38
#define __FLT_DECIMAL_DIG__ 9
#define __FLT_MAX__ 3.40282346638528859811704183484516925e+38F
#define __FLT_MIN__ 1.17549435082228750796873653722224568e-38F
#define __FLT_EPSILON__ 1.19209289550781250000000000000000000e-7F
#define __FLT_DENORM_MIN__ 1.40129846432481707092372958328991613e-45F
#define __FLT_HAS_DENORM__ 1
#define __FLT_HAS_INFINITY__ 1
#define __FLT_HAS_QUIET_NAN__ 1
#define __DBL_MANT_DIG__ 53
#define __DBL_DIG__ 15
#define __DBL_MIN_EXP__ (-1021)
#define __DBL_MIN_10_EXP__ (-307)
#define __DBL_MAX_EXP__ 1024
#define __DBL_MAX_10_EXP__ 308
#define __DBL_DECIMAL_DIG__ 17
#define __DBL_MAX__ ((double)1.79769313486231570814527423731704357e+308L)
#define __DBL_MIN__ ((double)2.22507385850720138309023271733240406e-308L)
#define __DBL_EPSILON__ ((double)2.22044604925031308084726333618164062e-16L)
#define __DBL_DENORM_MIN__ ((double)4.94065645841246544176568792868221372e-324L)
#define __DBL_HAS_DENORM__ 1
#define __DBL_HAS_INFINITY__ 1
#define __DBL_HAS_QUIET_NAN__ 1
#define __LDBL_MANT_DIG__ 64
#define __LDBL_DIG__ 18
#define __LDBL_MIN_EXP__ (-16381)
#define __LDBL_MIN_10_EXP__ (-4931)
#define __LDBL_MAX_EXP__ 16384
#define __LDBL_MAX_10_EXP__ 4932
#define __DECIMAL_DIG__ 21
#define __LDBL_DECIMAL_DIG__ 21
#define __LDBL_MAX__ 1.18973149535723176502126385303097021e+4932L
#define __LDBL_MIN__ 3.36210314311209350626267781732175260e-4932L
#define __LDBL_EPSILON__ 1.08420217248550443400745280086994171e-19L
#define __LDBL_DENORM_MIN__ 3.64519953188247460252840593361941982e-4951L
#define __LDBL_HAS_DENORM__ 1
#define __LDBL_HAS_INFINITY__ 1
#define __LDBL_HAS_QUIET_NAN__ 1
#define __FLT32_MANT_DIG__ 24
#define __FLT32_DIG__ 6
#define __FLT32_MIN_EXP__ (-125)
#define __FLT32_MIN_10_EXP__ (-37)
#define __FLT32_MAX_EXP__ 128
#define __FLT32_MAX_10_EXP__ 38
#define __FLT32_DECIMAL_DIG__ 9
#define __FLT32_MAX__ 3.40282346638528859811704183484516925e+38F32
#define __FLT32_MIN__ 1.17549435082228750796873653722224568e-38F32
#define __FLT32_EPSILON__ 1.19209289550781250000000000000000000e-7F32
#define __FLT32_DENORM_MIN__ 1.40129846432481707092372958328991613e-45F32
#define __FLT32_HAS_DENORM__ 1
#define __FLT32_HAS_INFINITY__ 1
#define __FLT32_HAS_QUIET_NAN__ 1
#define __FLT64_MANT_DIG__ 53
#define __FLT64_DIG__ 15
#define __FLT64_MIN_EXP__ (-1021)
#define __FLT64_MIN_10_EXP__ (-307)
#define __FLT64_MAX_EXP__ 1024
#define __FLT64_MAX_10_EXP__ 308
#define __FLT64_DECIMAL_DIG__ 17
#define __FLT64_MAX__ 1.79769313486231570814527423731704357e+308F64
#define __FLT64_MIN__ 2.22507385850720138309023271733240406e-308F64
#define __FLT64_EPSILON__ 2.22044604925031308084726333618164062e-16F64
#define __FLT64_DENORM_MIN__ 4.94065645841246544176568792868221372e-324F64
#define __FLT64_HAS_DENORM__ 1
#define __FLT64_HAS_INFINITY__ 1
#define __FLT64_HAS_QUIET_NAN__ 1
#define __FLT128_MANT_DIG__ 113
#define __FLT128_DIG__ 33
#define __FLT128_MIN_EXP__ (-16381)
#define __FLT128_MIN_10_EXP__ (-4931)
#define __FLT128_MAX_EXP__ 16384
#define __FLT128_MAX_10_EXP__ 4932
#define __FLT128_DECIMAL_DIG__ 36
#define __FLT128_MAX__ 1.18973149535723176508575932662800702e+4932F128
#define __FLT128_MIN__ 3.36210314311209350626267781732175260e-4932F128
#define __FLT128_EPSILON__ 1.92592994438723585305597794258492732e-34F128
#define __FLT128_DENORM_MIN__ 6.47517511943802511092443895822764655e-4966F128
#define __FLT128_HAS_DENORM__ 1
#define __FLT128_HAS_INFINITY__ 1
#define __FLT128_HAS_QUIET_NAN__ 1
#define __FLT32X_MANT_DIG__ 53
#define __FLT32X_DIG__ 15
#define __FLT32X_MIN_EXP__ (-1021)
#define __FLT32X_MIN_10_EXP__ (-307)
#define __FLT32X_MAX_EXP__ 1024
#define __FLT32X_MAX_10_EXP__ 308
#define __FLT32X_DECIMAL_DIG__ 17
#define __FLT32X_MAX__ 1.79769313486231570814527423731704357e+308F32x
#define __FLT32X_MIN__ 2.22507385850720138309023271733240406e-308F32x
#define __FLT32X_EPSILON__ 2.22044604925031308084726333618164062e-16F32x
#define __FLT32X_DENORM_MIN__ 4.94065645841246544176568792868221372e-324F32x
#define __FLT32X_HAS_DENORM__ 1
#define __FLT32X_HAS_INFINITY__ 1
#define __FLT32X_HAS_QUIET_NAN__ 1
#define __FLT64X_MANT_DIG__ 64
#define __FLT64X_DIG__ 18
#define __FLT64X_MIN_EXP__ (-16381)
#define __FLT64X_MIN_10_EXP__ (-4931)
#define __FLT64X_MAX_EXP__ 16384
#define __FLT64X_MAX_10_EXP__ 4932
#define __FLT64X_DECIMAL_DIG__ 21
#define __FLT64X_MAX__ 1.18973149535723176502126385303097021e+4932F64x
#define __FLT64X_MIN__ 3.36210314311209350626267781732175260e-4932F64x
#define __FLT64X_EPSILON__ 1.08420217248550443400745280086994171e-19F64x
#define __FLT64X_DENORM_MIN__ 3.64519953188247460252840593361941982e-4951F64x
#define __FLT64X_HAS_DENORM__ 1
#define __FLT64X_HAS_INFINITY__ 1
#define __FLT64X_HAS_QUIET_NAN__ 1
#define __DEC32_MANT_DIG__ 7
#define __DEC32_MIN_EXP__ (-94)
#define __DEC32_MAX_EXP__ 97
#define __DEC32_MIN__ 1E-95DF
#define __DEC32_MAX__ 9.999999E96DF
#define __DEC32_EPSILON__ 1E-6DF
#define __DEC32_SUBNORMAL_MIN__ 0.000001E-95DF
#define __DEC64_MANT_DIG__ 16
#define __DEC64_MIN_EXP__ (-382)
#define __DEC64_MAX_EXP__ 385
#define __DEC64_MIN__ 1E-383DD
#define __DEC64_MAX__ 9.999999999999999E384DD
#define __DEC64_EPSILON__ 1E-15DD
#define __DEC64_SUBNORMAL_MIN__ 0.000000000000001E-383DD
#define __DEC128_MANT_DIG__ 34
#define __DEC128_MIN_EXP__ (-6142)
#define __DEC128_MAX_EXP__ 6145
#define __DEC128_MIN__ 1E-6143DL
#define __DEC128_MAX__ 9.999999999999999999999999999999999E6144DL
#define __DEC128_EPSILON__ 1E-33DL
#define __DEC128_SUBNORMAL_MIN__ 0.000000000000000000000000000000001E-6143DL
#define __REGISTER_PREFIX__ 
#define __USER_LABEL_PREFIX__ 
#define __GNUC_STDC_INLINE__ 1
#define __NO_INLINE__ 1
#define __STRICT_ANSI__ 1
#define __GCC_HAVE_SYNC_COMPARE_AND_SWAP_1 1
#define __GCC_HAVE_SYNC_COMPARE_AND_SWAP_2 1
#define __GCC_HAVE_SYNC_COMPARE_AND_SWAP_4 1
#define __GCC_HAVE_SYNC_COMPARE_AND_SWAP_8 1
#define __GCC_ATOMIC_BOOL_LOCK_FREE 2
#define __GCC_ATOMIC_CHAR_LOCK_FREE 2
#define __GCC_ATOMIC_CHAR16_T_LOCK_FREE 2
#define __GCC_ATOMIC_CHAR32_T_LOCK_FREE 2
#define __GCC_ATOMIC_WCHAR_T_LOCK_FREE 2
#define __GCC_ATOMIC_SHORT_LOCK_FREE 2
#define __GCC_ATOMIC_INT_LOCK_FREE 2
#define __GCC_ATOMIC_LONG_LOCK_FREE 2
#define __GCC_ATOMIC_LLONG_LOCK_FREE 2
#define __GCC_ATOMIC_TEST_AND_SET_TRUEVAL 1
#define __GCC_ATOMIC_POINTER_LOCK_FREE 2
#define __GCC_HAVE_DWARF2_CFI_ASM 1
#define __PRAGMA_REDEFINE_EXTNAME 1
#define __SSP_STRONG__ 3
#define __SIZEOF_INT128__ 16
#define __SIZEOF_WCHAR_T__ 4
#define __SIZEOF_WINT_T__ 4
#define __SIZEOF_PTRDIFF_T__ 8
#define __amd64 1
#define __amd64__ 1
#define __x86_64 1
#define __x86_64__ 1
#define __SIZEOF_FLOAT80__ 16
#define __SIZEOF_FLOAT128__ 16
#define __ATOMIC_HLE_ACQUIRE 65536
#define __ATOMIC_HLE_RELEASE 131072
#define __GCC_ASM_FLAG_OUTPUTS__ 1
#define __k8 1
#define __k8__ 1
#define __code_model_small__ 1
#define __MMX__ 1
#define __SSE__ 1
#define __SSE2__ 1
#define __FXSR__ 1
#define __SSE_MATH__ 1
#define __SSE2_MATH__ 1
#define __SEG_FS 1
#define __SEG_GS 1
#define __gnu_linux__ 1
#define __linux 1
#define __linux__ 1
#define __unix 1
#define __unix__ 1
#define __ELF__ 1
#define __DECIMAL_BID_FORMAT__ 1
# 1 "<command-line>"
# 31 "<command-line>"
# 1 "/usr/include/stdc-predef.h" 1 3 4
# 19 "/usr/include/stdc-predef.h" 3 4
#define _STDC_PREDEF_H 1
# 38 "/usr/include/stdc-predef.h" 3 4
#define __STDC_IEC_559__ 1







#define __STDC_IEC_559_COMPLEX__ 1
# 58 "/usr/include/stdc-predef.h" 3 4
#define __STDC_ISO_10646__ 201706L


#define __STDC_NO_THREADS__ 1
# 32 "<command-line>" 2
# 1 "checkinput.c"

# 1 "checkinput.h" 1







#define _CHECKINPUT_H_ 
void checkInput(int err);
# 3 "checkinput.c" 2
# 1 "/usr/include/stdio.h" 1 3 4
# 24 "/usr/include/stdio.h" 3 4
#define _STDIO_H 1

#define __GLIBC_INTERNAL_STARTING_HEADER_IMPLEMENTATION 
# 1 "/usr/include/x86_64-linux-gnu/bits/libc-header-start.h" 1 3 4
# 31 "/usr/include/x86_64-linux-gnu/bits/libc-header-start.h" 3 4
#undef __GLIBC_INTERNAL_STARTING_HEADER_IMPLEMENTATION

# 1 "/usr/include/features.h" 1 3 4
# 19 "/usr/include/features.h" 3 4
#define _FEATURES_H 1
# 119 "/usr/include/features.h" 3 4
#undef __USE_ISOC11
#undef __USE_ISOC99
#undef __USE_ISOC95
#undef __USE_ISOCXX11
#undef __USE_POSIX
#undef __USE_POSIX2
#undef __USE_POSIX199309
#undef __USE_POSIX199506
#undef __USE_XOPEN
#undef __USE_XOPEN_EXTENDED
#undef __USE_UNIX98
#undef __USE_XOPEN2K
#undef __USE_XOPEN2KXSI
#undef __USE_XOPEN2K8
#undef __USE_XOPEN2K8XSI
#undef __USE_LARGEFILE
#undef __USE_LARGEFILE64
#undef __USE_FILE_OFFSET64
#undef __USE_MISC
#undef __USE_ATFILE
#undef __USE_GNU
#undef __USE_FORTIFY_LEVEL
#undef __KERNEL_STRICT_NAMES
#undef __GLIBC_USE_DEPRECATED_GETS




#define __KERNEL_STRICT_NAMES 
# 158 "/usr/include/features.h" 3 4
#define __GNUC_PREREQ(maj,min) ((__GNUC__ << 16) + __GNUC_MINOR__ >= ((maj) << 16) + (min))
# 172 "/usr/include/features.h" 3 4
#define __glibc_clang_prereq(maj,min) 0



#define __GLIBC_USE(F) __GLIBC_USE_ ## F
# 233 "/usr/include/features.h" 3 4
#define __USE_ISOC99 1





#define __USE_ISOC95 1
# 387 "/usr/include/features.h" 3 4
#define __USE_FORTIFY_LEVEL 0
# 397 "/usr/include/features.h" 3 4
#define __GLIBC_USE_DEPRECATED_GETS 1
# 410 "/usr/include/features.h" 3 4
#undef __GNU_LIBRARY__
#define __GNU_LIBRARY__ 6



#define __GLIBC__ 2
#define __GLIBC_MINOR__ 27

#define __GLIBC_PREREQ(maj,min) ((__GLIBC__ << 16) + __GLIBC_MINOR__ >= ((maj) << 16) + (min))





# 1 "/usr/include/x86_64-linux-gnu/sys/cdefs.h" 1 3 4
# 19 "/usr/include/x86_64-linux-gnu/sys/cdefs.h" 3 4
#define _SYS_CDEFS_H 1
# 34 "/usr/include/x86_64-linux-gnu/sys/cdefs.h" 3 4
#undef __P
#undef __PMT






#define __LEAF , __leaf__
#define __LEAF_ATTR __attribute__ ((__leaf__))
# 55 "/usr/include/x86_64-linux-gnu/sys/cdefs.h" 3 4
#define __THROW __attribute__ ((__nothrow__ __LEAF))
#define __THROWNL __attribute__ ((__nothrow__))
#define __NTH(fct) __attribute__ ((__nothrow__ __LEAF)) fct
#define __NTHNL(fct) __attribute__ ((__nothrow__)) fct
# 89 "/usr/include/x86_64-linux-gnu/sys/cdefs.h" 3 4
#define __glibc_clang_has_extension(ext) 0




#define __P(args) args
#define __PMT(args) args




#define __CONCAT(x,y) x ## y
#define __STRING(x) #x


#define __ptr_t void *







#define __BEGIN_DECLS 
#define __END_DECLS 




#define __bos(ptr) __builtin_object_size (ptr, __USE_FORTIFY_LEVEL > 1)
#define __bos0(ptr) __builtin_object_size (ptr, 0)


#define __warndecl(name,msg) extern void name (void) __attribute__((__warning__ (msg)))

#define __warnattr(msg) __attribute__((__warning__ (msg)))
#define __errordecl(name,msg) extern void name (void) __attribute__((__error__ (msg)))
# 138 "/usr/include/x86_64-linux-gnu/sys/cdefs.h" 3 4
#define __flexarr []
#define __glibc_c99_flexarr_available 1
# 169 "/usr/include/x86_64-linux-gnu/sys/cdefs.h" 3 4
#define __REDIRECT(name,proto,alias) name proto __asm__ (__ASMNAME (#alias))






#define __REDIRECT_NTH(name,proto,alias) name proto __asm__ (__ASMNAME (#alias)) __THROW

#define __REDIRECT_NTHNL(name,proto,alias) name proto __asm__ (__ASMNAME (#alias)) __THROWNL


#define __ASMNAME(cname) __ASMNAME2 (__USER_LABEL_PREFIX__, cname)
#define __ASMNAME2(prefix,cname) __STRING (prefix) cname
# 203 "/usr/include/x86_64-linux-gnu/sys/cdefs.h" 3 4
#define __attribute_malloc__ __attribute__ ((__malloc__))







#define __attribute_alloc_size__(params) __attribute__ ((__alloc_size__ params))
# 221 "/usr/include/x86_64-linux-gnu/sys/cdefs.h" 3 4
#define __attribute_pure__ __attribute__ ((__pure__))






#define __attribute_const__ __attribute__ ((__const__))
# 237 "/usr/include/x86_64-linux-gnu/sys/cdefs.h" 3 4
#define __attribute_used__ __attribute__ ((__used__))
#define __attribute_noinline__ __attribute__ ((__noinline__))







#define __attribute_deprecated__ __attribute__ ((__deprecated__))
# 256 "/usr/include/x86_64-linux-gnu/sys/cdefs.h" 3 4
#define __attribute_deprecated_msg__(msg) __attribute__ ((__deprecated__ (msg)))
# 269 "/usr/include/x86_64-linux-gnu/sys/cdefs.h" 3 4
#define __attribute_format_arg__(x) __attribute__ ((__format_arg__ (x)))
# 279 "/usr/include/x86_64-linux-gnu/sys/cdefs.h" 3 4
#define __attribute_format_strfmon__(a,b) __attribute__ ((__format__ (__strfmon__, a, b)))
# 288 "/usr/include/x86_64-linux-gnu/sys/cdefs.h" 3 4
#define __nonnull(params) __attribute__ ((__nonnull__ params))







#define __attribute_warn_unused_result__ __attribute__ ((__warn_unused_result__))
# 305 "/usr/include/x86_64-linux-gnu/sys/cdefs.h" 3 4
#define __wur 







#undef __always_inline
#define __always_inline __inline __attribute__ ((__always_inline__))
# 323 "/usr/include/x86_64-linux-gnu/sys/cdefs.h" 3 4
#define __attribute_artificial__ __attribute__ ((__artificial__))
# 341 "/usr/include/x86_64-linux-gnu/sys/cdefs.h" 3 4
#define __extern_inline extern __inline __attribute__ ((__gnu_inline__))
#define __extern_always_inline extern __always_inline __attribute__ ((__gnu_inline__))
# 351 "/usr/include/x86_64-linux-gnu/sys/cdefs.h" 3 4
#define __fortify_function __extern_always_inline __attribute_artificial__





#define __va_arg_pack() __builtin_va_arg_pack ()
#define __va_arg_pack_len() __builtin_va_arg_pack_len ()
# 378 "/usr/include/x86_64-linux-gnu/sys/cdefs.h" 3 4
#define __restrict_arr __restrict
# 393 "/usr/include/x86_64-linux-gnu/sys/cdefs.h" 3 4
#define __glibc_unlikely(cond) __builtin_expect ((cond), 0)
#define __glibc_likely(cond) __builtin_expect ((cond), 1)
# 416 "/usr/include/x86_64-linux-gnu/sys/cdefs.h" 3 4
#define __attribute_nonstring__ 





#define _Static_assert(expr,diagnostic) extern int (*__Static_assert_function (void)) [!!sizeof (struct { int __error_if_negative: (expr) ? 2 : -1; })]




# 1 "/usr/include/x86_64-linux-gnu/bits/wordsize.h" 1 3 4



#define __WORDSIZE 64







#define __WORDSIZE_TIME64_COMPAT32 1

#define __SYSCALL_WORDSIZE 64
# 428 "/usr/include/x86_64-linux-gnu/sys/cdefs.h" 2 3 4
# 1 "/usr/include/x86_64-linux-gnu/bits/long-double.h" 1 3 4
# 429 "/usr/include/x86_64-linux-gnu/sys/cdefs.h" 2 3 4
# 450 "/usr/include/x86_64-linux-gnu/sys/cdefs.h" 3 4
#define __LDBL_REDIR1(name,proto,alias) name proto
#define __LDBL_REDIR(name,proto) name proto
#define __LDBL_REDIR1_NTH(name,proto,alias) name proto __THROW
#define __LDBL_REDIR_NTH(name,proto) name proto __THROW
#define __LDBL_REDIR_DECL(name) 

#define __REDIRECT_LDBL(name,proto,alias) __REDIRECT (name, proto, alias)
#define __REDIRECT_NTH_LDBL(name,proto,alias) __REDIRECT_NTH (name, proto, alias)
# 468 "/usr/include/x86_64-linux-gnu/sys/cdefs.h" 3 4
#define __glibc_macro_warning1(message) _Pragma (#message)
#define __glibc_macro_warning(message) __glibc_macro_warning1 (GCC warning message)
# 487 "/usr/include/x86_64-linux-gnu/sys/cdefs.h" 3 4
#define __HAVE_GENERIC_SELECTION 1
# 425 "/usr/include/features.h" 2 3 4
# 448 "/usr/include/features.h" 3 4
# 1 "/usr/include/x86_64-linux-gnu/gnu/stubs.h" 1 3 4
# 10 "/usr/include/x86_64-linux-gnu/gnu/stubs.h" 3 4
# 1 "/usr/include/x86_64-linux-gnu/gnu/stubs-64.h" 1 3 4
# 10 "/usr/include/x86_64-linux-gnu/gnu/stubs-64.h" 3 4
#define __stub___compat_bdflush 
#define __stub_chflags 
#define __stub_fattach 
#define __stub_fchflags 
#define __stub_fdetach 
#define __stub_getmsg 
#define __stub_gtty 
#define __stub_lchmod 
#define __stub_putmsg 
#define __stub_revoke 
#define __stub_setlogin 
#define __stub_sigreturn 
#define __stub_sstk 
#define __stub_stty 
# 11 "/usr/include/x86_64-linux-gnu/gnu/stubs.h" 2 3 4
# 449 "/usr/include/features.h" 2 3 4
# 34 "/usr/include/x86_64-linux-gnu/bits/libc-header-start.h" 2 3 4



#undef __GLIBC_USE_LIB_EXT2




#define __GLIBC_USE_LIB_EXT2 0




#undef __GLIBC_USE_IEC_60559_BFP_EXT



#define __GLIBC_USE_IEC_60559_BFP_EXT 0




#undef __GLIBC_USE_IEC_60559_FUNCS_EXT



#define __GLIBC_USE_IEC_60559_FUNCS_EXT 0




#undef __GLIBC_USE_IEC_60559_TYPES_EXT



#define __GLIBC_USE_IEC_60559_TYPES_EXT 0
# 28 "/usr/include/stdio.h" 2 3 4



#define __need_size_t 
#define __need_NULL 
# 1 "/usr/lib/gcc/x86_64-linux-gnu/7/include/stddef.h" 1 3 4
# 187 "/usr/lib/gcc/x86_64-linux-gnu/7/include/stddef.h" 3 4
#define __size_t__ 
#define __SIZE_T__ 
#define _SIZE_T 
#define _SYS_SIZE_T_H 
#define _T_SIZE_ 
#define _T_SIZE 
#define __SIZE_T 
#define _SIZE_T_ 
#define _BSD_SIZE_T_ 
#define _SIZE_T_DEFINED_ 
#define _SIZE_T_DEFINED 
#define _BSD_SIZE_T_DEFINED_ 
#define _SIZE_T_DECLARED 
#define ___int_size_t_h 
#define _GCC_SIZE_T 
#define _SIZET_ 







#define __size_t 






# 216 "/usr/lib/gcc/x86_64-linux-gnu/7/include/stddef.h" 3 4
typedef long unsigned int size_t;
# 238 "/usr/lib/gcc/x86_64-linux-gnu/7/include/stddef.h" 3 4
#undef __need_size_t
# 401 "/usr/lib/gcc/x86_64-linux-gnu/7/include/stddef.h" 3 4
#undef NULL




#define NULL ((void *)0)





#undef __need_NULL
# 34 "/usr/include/stdio.h" 2 3 4

# 1 "/usr/include/x86_64-linux-gnu/bits/types.h" 1 3 4
# 24 "/usr/include/x86_64-linux-gnu/bits/types.h" 3 4
#define _BITS_TYPES_H 1


# 1 "/usr/include/x86_64-linux-gnu/bits/wordsize.h" 1 3 4



#define __WORDSIZE 64







#define __WORDSIZE_TIME64_COMPAT32 1

#define __SYSCALL_WORDSIZE 64
# 28 "/usr/include/x86_64-linux-gnu/bits/types.h" 2 3 4


typedef unsigned char __u_char;
typedef unsigned short int __u_short;
typedef unsigned int __u_int;
typedef unsigned long int __u_long;


typedef signed char __int8_t;
typedef unsigned char __uint8_t;
typedef signed short int __int16_t;
typedef unsigned short int __uint16_t;
typedef signed int __int32_t;
typedef unsigned int __uint32_t;

typedef signed long int __int64_t;
typedef unsigned long int __uint64_t;







typedef long int __quad_t;
typedef unsigned long int __u_quad_t;







typedef long int __intmax_t;
typedef unsigned long int __uintmax_t;
# 98 "/usr/include/x86_64-linux-gnu/bits/types.h" 3 4
#define __S16_TYPE short int
#define __U16_TYPE unsigned short int
#define __S32_TYPE int
#define __U32_TYPE unsigned int
#define __SLONGWORD_TYPE long int
#define __ULONGWORD_TYPE unsigned long int
# 117 "/usr/include/x86_64-linux-gnu/bits/types.h" 3 4
#define __SQUAD_TYPE long int
#define __UQUAD_TYPE unsigned long int
#define __SWORD_TYPE long int
#define __UWORD_TYPE unsigned long int
#define __SLONG32_TYPE int
#define __ULONG32_TYPE unsigned int
#define __S64_TYPE long int
#define __U64_TYPE unsigned long int

#define __STD_TYPE typedef



# 1 "/usr/include/x86_64-linux-gnu/bits/typesizes.h" 1 3 4
# 24 "/usr/include/x86_64-linux-gnu/bits/typesizes.h" 3 4
#define _BITS_TYPESIZES_H 1
# 34 "/usr/include/x86_64-linux-gnu/bits/typesizes.h" 3 4
#define __SYSCALL_SLONG_TYPE __SLONGWORD_TYPE
#define __SYSCALL_ULONG_TYPE __ULONGWORD_TYPE


#define __DEV_T_TYPE __UQUAD_TYPE
#define __UID_T_TYPE __U32_TYPE
#define __GID_T_TYPE __U32_TYPE
#define __INO_T_TYPE __SYSCALL_ULONG_TYPE
#define __INO64_T_TYPE __UQUAD_TYPE
#define __MODE_T_TYPE __U32_TYPE

#define __NLINK_T_TYPE __SYSCALL_ULONG_TYPE
#define __FSWORD_T_TYPE __SYSCALL_SLONG_TYPE




#define __OFF_T_TYPE __SYSCALL_SLONG_TYPE
#define __OFF64_T_TYPE __SQUAD_TYPE
#define __PID_T_TYPE __S32_TYPE
#define __RLIM_T_TYPE __SYSCALL_ULONG_TYPE
#define __RLIM64_T_TYPE __UQUAD_TYPE
#define __BLKCNT_T_TYPE __SYSCALL_SLONG_TYPE
#define __BLKCNT64_T_TYPE __SQUAD_TYPE
#define __FSBLKCNT_T_TYPE __SYSCALL_ULONG_TYPE
#define __FSBLKCNT64_T_TYPE __UQUAD_TYPE
#define __FSFILCNT_T_TYPE __SYSCALL_ULONG_TYPE
#define __FSFILCNT64_T_TYPE __UQUAD_TYPE
#define __ID_T_TYPE __U32_TYPE
#define __CLOCK_T_TYPE __SYSCALL_SLONG_TYPE
#define __TIME_T_TYPE __SYSCALL_SLONG_TYPE
#define __USECONDS_T_TYPE __U32_TYPE
#define __SUSECONDS_T_TYPE __SYSCALL_SLONG_TYPE
#define __DADDR_T_TYPE __S32_TYPE
#define __KEY_T_TYPE __S32_TYPE
#define __CLOCKID_T_TYPE __S32_TYPE
#define __TIMER_T_TYPE void *
#define __BLKSIZE_T_TYPE __SYSCALL_SLONG_TYPE
#define __FSID_T_TYPE struct { int __val[2]; }
#define __SSIZE_T_TYPE __SWORD_TYPE
#define __CPU_MASK_TYPE __SYSCALL_ULONG_TYPE





#define __OFF_T_MATCHES_OFF64_T 1


#define __INO_T_MATCHES_INO64_T 1


#define __RLIM_T_MATCHES_RLIM64_T 1





#define __FD_SETSIZE 1024
# 131 "/usr/include/x86_64-linux-gnu/bits/types.h" 2 3 4


typedef unsigned long int __dev_t;
typedef unsigned int __uid_t;
typedef unsigned int __gid_t;
typedef unsigned long int __ino_t;
typedef unsigned long int __ino64_t;
typedef unsigned int __mode_t;
typedef unsigned long int __nlink_t;
typedef long int __off_t;
typedef long int __off64_t;
typedef int __pid_t;
typedef struct { int __val[2]; } __fsid_t;
typedef long int __clock_t;
typedef unsigned long int __rlim_t;
typedef unsigned long int __rlim64_t;
typedef unsigned int __id_t;
typedef long int __time_t;
typedef unsigned int __useconds_t;
typedef long int __suseconds_t;

typedef int __daddr_t;
typedef int __key_t;


typedef int __clockid_t;


typedef void * __timer_t;


typedef long int __blksize_t;




typedef long int __blkcnt_t;
typedef long int __blkcnt64_t;


typedef unsigned long int __fsblkcnt_t;
typedef unsigned long int __fsblkcnt64_t;


typedef unsigned long int __fsfilcnt_t;
typedef unsigned long int __fsfilcnt64_t;


typedef long int __fsword_t;

typedef long int __ssize_t;


typedef long int __syscall_slong_t;

typedef unsigned long int __syscall_ulong_t;



typedef __off64_t __loff_t;
typedef char *__caddr_t;


typedef long int __intptr_t;


typedef unsigned int __socklen_t;




typedef int __sig_atomic_t;

#undef __STD_TYPE
# 36 "/usr/include/stdio.h" 2 3 4
# 1 "/usr/include/x86_64-linux-gnu/bits/types/__FILE.h" 1 3 4

#define ____FILE_defined 1

struct _IO_FILE;
typedef struct _IO_FILE __FILE;
# 37 "/usr/include/stdio.h" 2 3 4
# 1 "/usr/include/x86_64-linux-gnu/bits/types/FILE.h" 1 3 4

#define __FILE_defined 1

struct _IO_FILE;


typedef struct _IO_FILE FILE;
# 38 "/usr/include/stdio.h" 2 3 4

#define _STDIO_USES_IOSTREAM 

# 1 "/usr/include/x86_64-linux-gnu/bits/libio.h" 1 3 4
# 29 "/usr/include/x86_64-linux-gnu/bits/libio.h" 3 4
#define _BITS_LIBIO_H 1





# 1 "/usr/include/x86_64-linux-gnu/bits/_G_config.h" 1 3 4




#define _BITS_G_CONFIG_H 1
# 14 "/usr/include/x86_64-linux-gnu/bits/_G_config.h" 3 4
#define __need_size_t 



#define __need_NULL 
# 1 "/usr/lib/gcc/x86_64-linux-gnu/7/include/stddef.h" 1 3 4
# 238 "/usr/lib/gcc/x86_64-linux-gnu/7/include/stddef.h" 3 4
#undef __need_size_t
# 401 "/usr/lib/gcc/x86_64-linux-gnu/7/include/stddef.h" 3 4
#undef NULL




#define NULL ((void *)0)





#undef __need_NULL
# 20 "/usr/include/x86_64-linux-gnu/bits/_G_config.h" 2 3 4

# 1 "/usr/include/x86_64-linux-gnu/bits/types/__mbstate_t.h" 1 3 4

#define ____mbstate_t_defined 1
# 13 "/usr/include/x86_64-linux-gnu/bits/types/__mbstate_t.h" 3 4
typedef struct
{
  int __count;
  union
  {
    unsigned int __wch;
    char __wchb[4];
  } __value;
} __mbstate_t;
# 22 "/usr/include/x86_64-linux-gnu/bits/_G_config.h" 2 3 4




typedef struct
{
  __off_t __pos;
  __mbstate_t __state;
} _G_fpos_t;
typedef struct
{
  __off64_t __pos;
  __mbstate_t __state;
} _G_fpos64_t;
# 51 "/usr/include/x86_64-linux-gnu/bits/_G_config.h" 3 4
#define _G_va_list __gnuc_va_list

#define _G_HAVE_MMAP 1
#define _G_HAVE_MREMAP 1

#define _G_IO_IO_FILE_VERSION 0x20001


#define _G_HAVE_ST_BLKSIZE defined (_STATBUF_ST_BLKSIZE)

#define _G_BUFSIZ 8192
# 36 "/usr/include/x86_64-linux-gnu/bits/libio.h" 2 3 4

#define _IO_fpos_t _G_fpos_t
#define _IO_fpos64_t _G_fpos64_t
#define _IO_size_t size_t
#define _IO_ssize_t __ssize_t
#define _IO_off_t __off_t
#define _IO_off64_t __off64_t
#define _IO_pid_t __pid_t
#define _IO_uid_t __uid_t
#define _IO_iconv_t _G_iconv_t
#define _IO_HAVE_ST_BLKSIZE _G_HAVE_ST_BLKSIZE
#define _IO_BUFSIZ _G_BUFSIZ
#define _IO_va_list _G_va_list
#define _IO_wint_t wint_t


#define __need___va_list 
# 1 "/usr/lib/gcc/x86_64-linux-gnu/7/include/stdarg.h" 1 3 4
# 34 "/usr/lib/gcc/x86_64-linux-gnu/7/include/stdarg.h" 3 4
#undef __need___va_list




#define __GNUC_VA_LIST 
typedef __builtin_va_list __gnuc_va_list;
# 54 "/usr/include/x86_64-linux-gnu/bits/libio.h" 2 3 4

#undef _IO_va_list
#define _IO_va_list __gnuc_va_list






#define _IO_UNIFIED_JUMPTABLES 1


#define EOF (-1)
# 81 "/usr/include/x86_64-linux-gnu/bits/libio.h" 3 4
#define _IOS_INPUT 1
#define _IOS_OUTPUT 2
#define _IOS_ATEND 4
#define _IOS_APPEND 8
#define _IOS_TRUNC 16
#define _IOS_NOCREATE 32
#define _IOS_NOREPLACE 64
#define _IOS_BIN 128







#define _IO_MAGIC 0xFBAD0000
#define _OLD_STDIO_MAGIC 0xFABC0000
#define _IO_MAGIC_MASK 0xFFFF0000
#define _IO_USER_BUF 1
#define _IO_UNBUFFERED 2
#define _IO_NO_READS 4
#define _IO_NO_WRITES 8
#define _IO_EOF_SEEN 0x10
#define _IO_ERR_SEEN 0x20
#define _IO_DELETE_DONT_CLOSE 0x40
#define _IO_LINKED 0x80
#define _IO_IN_BACKUP 0x100
#define _IO_LINE_BUF 0x200
#define _IO_TIED_PUT_GET 0x400
#define _IO_CURRENTLY_PUTTING 0x800
#define _IO_IS_APPENDING 0x1000
#define _IO_IS_FILEBUF 0x2000
#define _IO_BAD_SEEN 0x4000
#define _IO_USER_LOCK 0x8000

#define _IO_FLAGS2_MMAP 1
#define _IO_FLAGS2_NOTCANCEL 2



#define _IO_FLAGS2_USER_WBUF 8
# 130 "/usr/include/x86_64-linux-gnu/bits/libio.h" 3 4
#define _IO_SKIPWS 01
#define _IO_LEFT 02
#define _IO_RIGHT 04
#define _IO_INTERNAL 010
#define _IO_DEC 020
#define _IO_OCT 040
#define _IO_HEX 0100
#define _IO_SHOWBASE 0200
#define _IO_SHOWPOINT 0400
#define _IO_UPPERCASE 01000
#define _IO_SHOWPOS 02000
#define _IO_SCIENTIFIC 04000
#define _IO_FIXED 010000
#define _IO_UNITBUF 020000
#define _IO_STDIO 040000
#define _IO_DONT_CLOSE 0100000
#define _IO_BOOLALPHA 0200000


struct _IO_jump_t; struct _IO_FILE;




typedef void _IO_lock_t;





struct _IO_marker {
  struct _IO_marker *_next;
  struct _IO_FILE *_sbuf;



  int _pos;
# 177 "/usr/include/x86_64-linux-gnu/bits/libio.h" 3 4
};


enum __codecvt_result
{
  __codecvt_ok,
  __codecvt_partial,
  __codecvt_error,
  __codecvt_noconv
};
# 245 "/usr/include/x86_64-linux-gnu/bits/libio.h" 3 4
struct _IO_FILE {
  int _flags;
#define _IO_file_flags _flags



  char* _IO_read_ptr;
  char* _IO_read_end;
  char* _IO_read_base;
  char* _IO_write_base;
  char* _IO_write_ptr;
  char* _IO_write_end;
  char* _IO_buf_base;
  char* _IO_buf_end;

  char *_IO_save_base;
  char *_IO_backup_base;
  char *_IO_save_end;

  struct _IO_marker *_markers;

  struct _IO_FILE *_chain;

  int _fileno;



  int _flags2;

  __off_t _old_offset;

#define __HAVE_COLUMN 

  unsigned short _cur_column;
  signed char _vtable_offset;
  char _shortbuf[1];



  _IO_lock_t *_lock;
# 293 "/usr/include/x86_64-linux-gnu/bits/libio.h" 3 4
  __off64_t _offset;







  void *__pad1;
  void *__pad2;
  void *__pad3;
  void *__pad4;

  size_t __pad5;
  int _mode;

  char _unused2[15 * sizeof (int) - 4 * sizeof (void *) - sizeof (size_t)];

};


typedef struct _IO_FILE _IO_FILE;


struct _IO_FILE_plus;

extern struct _IO_FILE_plus _IO_2_1_stdin_;
extern struct _IO_FILE_plus _IO_2_1_stdout_;
extern struct _IO_FILE_plus _IO_2_1_stderr_;

#define _IO_stdin ((_IO_FILE*)(&_IO_2_1_stdin_))
#define _IO_stdout ((_IO_FILE*)(&_IO_2_1_stdout_))
#define _IO_stderr ((_IO_FILE*)(&_IO_2_1_stderr_))
# 337 "/usr/include/x86_64-linux-gnu/bits/libio.h" 3 4
typedef __ssize_t __io_read_fn (void *__cookie, char *__buf, size_t __nbytes);







typedef __ssize_t __io_write_fn (void *__cookie, const char *__buf,
     size_t __n);







typedef int __io_seek_fn (void *__cookie, __off64_t *__pos, int __w);


typedef int __io_close_fn (void *__cookie);
# 389 "/usr/include/x86_64-linux-gnu/bits/libio.h" 3 4
extern int __underflow (_IO_FILE *);
extern int __uflow (_IO_FILE *);
extern int __overflow (_IO_FILE *, int);







#define _IO_BE(expr,res) __builtin_expect ((expr), res)




#define _IO_getc_unlocked(_fp) (_IO_BE ((_fp)->_IO_read_ptr >= (_fp)->_IO_read_end, 0) ? __uflow (_fp) : *(unsigned char *) (_fp)->_IO_read_ptr++)


#define _IO_peekc_unlocked(_fp) (_IO_BE ((_fp)->_IO_read_ptr >= (_fp)->_IO_read_end, 0) && __underflow (_fp) == EOF ? EOF : *(unsigned char *) (_fp)->_IO_read_ptr)



#define _IO_putc_unlocked(_ch,_fp) (_IO_BE ((_fp)->_IO_write_ptr >= (_fp)->_IO_write_end, 0) ? __overflow (_fp, (unsigned char) (_ch)) : (unsigned char) (*(_fp)->_IO_write_ptr++ = (_ch)))
# 430 "/usr/include/x86_64-linux-gnu/bits/libio.h" 3 4
#define _IO_feof_unlocked(__fp) (((__fp)->_flags & _IO_EOF_SEEN) != 0)
#define _IO_ferror_unlocked(__fp) (((__fp)->_flags & _IO_ERR_SEEN) != 0)

extern int _IO_getc (_IO_FILE *__fp);
extern int _IO_putc (int __c, _IO_FILE *__fp);
extern int _IO_feof (_IO_FILE *__fp) __attribute__ ((__nothrow__ , __leaf__));
extern int _IO_ferror (_IO_FILE *__fp) __attribute__ ((__nothrow__ , __leaf__));

extern int _IO_peekc_locked (_IO_FILE *__fp);


#define _IO_PENDING_OUTPUT_COUNT(_fp) ((_fp)->_IO_write_ptr - (_fp)->_IO_write_base)


extern void _IO_flockfile (_IO_FILE *) __attribute__ ((__nothrow__ , __leaf__));
extern void _IO_funlockfile (_IO_FILE *) __attribute__ ((__nothrow__ , __leaf__));
extern int _IO_ftrylockfile (_IO_FILE *) __attribute__ ((__nothrow__ , __leaf__));

#define _IO_peekc(_fp) _IO_peekc_unlocked (_fp)
#define _IO_flockfile(_fp) 
#define _IO_funlockfile(_fp) 
#define _IO_ftrylockfile(_fp) 

#define _IO_cleanup_region_start(_fct,_fp) 


#define _IO_cleanup_region_end(_Doit) 


#define _IO_need_lock(_fp) (((_fp)->_flags2 & _IO_FLAGS2_NEED_LOCK) != 0)


extern int _IO_vfscanf (_IO_FILE * __restrict, const char * __restrict,
   __gnuc_va_list, int *__restrict);
extern int _IO_vfprintf (_IO_FILE *__restrict, const char *__restrict,
    __gnuc_va_list);
extern __ssize_t _IO_padn (_IO_FILE *, int, __ssize_t);
extern size_t _IO_sgetn (_IO_FILE *, void *, size_t);

extern __off64_t _IO_seekoff (_IO_FILE *, __off64_t, int, int);
extern __off64_t _IO_seekpos (_IO_FILE *, __off64_t, int);

extern void _IO_free_backup_area (_IO_FILE *) __attribute__ ((__nothrow__ , __leaf__));
# 42 "/usr/include/stdio.h" 2 3 4
# 78 "/usr/include/stdio.h" 3 4
typedef _G_fpos_t fpos_t;
# 87 "/usr/include/stdio.h" 3 4
#define _IOFBF 0
#define _IOLBF 1
#define _IONBF 2




#define BUFSIZ _IO_BUFSIZ
# 107 "/usr/include/stdio.h" 3 4
#define SEEK_SET 0
#define SEEK_CUR 1
#define SEEK_END 2
# 131 "/usr/include/stdio.h" 3 4
# 1 "/usr/include/x86_64-linux-gnu/bits/stdio_lim.h" 1 3 4
# 19 "/usr/include/x86_64-linux-gnu/bits/stdio_lim.h" 3 4
#define _BITS_STDIO_LIM_H 1





#define L_tmpnam 20
#define TMP_MAX 238328
#define FILENAME_MAX 4096
# 36 "/usr/include/x86_64-linux-gnu/bits/stdio_lim.h" 3 4
#undef FOPEN_MAX
#define FOPEN_MAX 16
# 132 "/usr/include/stdio.h" 2 3 4



extern struct _IO_FILE *stdin;
extern struct _IO_FILE *stdout;
extern struct _IO_FILE *stderr;

#define stdin stdin
#define stdout stdout
#define stderr stderr


extern int remove (const char *__filename) __attribute__ ((__nothrow__ , __leaf__));

extern int rename (const char *__old, const char *__new) __attribute__ ((__nothrow__ , __leaf__));
# 159 "/usr/include/stdio.h" 3 4
extern FILE *tmpfile (void) ;
# 173 "/usr/include/stdio.h" 3 4
extern char *tmpnam (char *__s) __attribute__ ((__nothrow__ , __leaf__)) ;
# 199 "/usr/include/stdio.h" 3 4
extern int fclose (FILE *__stream);




extern int fflush (FILE *__stream);
# 232 "/usr/include/stdio.h" 3 4
extern FILE *fopen (const char *__restrict __filename,
      const char *__restrict __modes) ;




extern FILE *freopen (const char *__restrict __filename,
        const char *__restrict __modes,
        FILE *__restrict __stream) ;
# 290 "/usr/include/stdio.h" 3 4
extern void setbuf (FILE *__restrict __stream, char *__restrict __buf) __attribute__ ((__nothrow__ , __leaf__));



extern int setvbuf (FILE *__restrict __stream, char *__restrict __buf,
      int __modes, size_t __n) __attribute__ ((__nothrow__ , __leaf__));
# 312 "/usr/include/stdio.h" 3 4
extern int fprintf (FILE *__restrict __stream,
      const char *__restrict __format, ...);




extern int printf (const char *__restrict __format, ...);

extern int sprintf (char *__restrict __s,
      const char *__restrict __format, ...) __attribute__ ((__nothrow__));





extern int vfprintf (FILE *__restrict __s, const char *__restrict __format,
       __gnuc_va_list __arg);




extern int vprintf (const char *__restrict __format, __gnuc_va_list __arg);

extern int vsprintf (char *__restrict __s, const char *__restrict __format,
       __gnuc_va_list __arg) __attribute__ ((__nothrow__));



extern int snprintf (char *__restrict __s, size_t __maxlen,
       const char *__restrict __format, ...)
     __attribute__ ((__nothrow__)) __attribute__ ((__format__ (__printf__, 3, 4)));

extern int vsnprintf (char *__restrict __s, size_t __maxlen,
        const char *__restrict __format, __gnuc_va_list __arg)
     __attribute__ ((__nothrow__)) __attribute__ ((__format__ (__printf__, 3, 0)));
# 377 "/usr/include/stdio.h" 3 4
extern int fscanf (FILE *__restrict __stream,
     const char *__restrict __format, ...) ;




extern int scanf (const char *__restrict __format, ...) ;

extern int sscanf (const char *__restrict __s,
     const char *__restrict __format, ...) __attribute__ ((__nothrow__ , __leaf__));
# 395 "/usr/include/stdio.h" 3 4
extern int fscanf (FILE *__restrict __stream, const char *__restrict __format, ...) __asm__ ("" "__isoc99_fscanf")

                               ;
extern int scanf (const char *__restrict __format, ...) __asm__ ("" "__isoc99_scanf")
                              ;
extern int sscanf (const char *__restrict __s, const char *__restrict __format, ...) __asm__ ("" "__isoc99_sscanf") __attribute__ ((__nothrow__ , __leaf__))

                      ;
# 420 "/usr/include/stdio.h" 3 4
extern int vfscanf (FILE *__restrict __s, const char *__restrict __format,
      __gnuc_va_list __arg)
     __attribute__ ((__format__ (__scanf__, 2, 0))) ;





extern int vscanf (const char *__restrict __format, __gnuc_va_list __arg)
     __attribute__ ((__format__ (__scanf__, 1, 0))) ;


extern int vsscanf (const char *__restrict __s,
      const char *__restrict __format, __gnuc_va_list __arg)
     __attribute__ ((__nothrow__ , __leaf__)) __attribute__ ((__format__ (__scanf__, 2, 0)));
# 443 "/usr/include/stdio.h" 3 4
extern int vfscanf (FILE *__restrict __s, const char *__restrict __format, __gnuc_va_list __arg) __asm__ ("" "__isoc99_vfscanf")



     __attribute__ ((__format__ (__scanf__, 2, 0))) ;
extern int vscanf (const char *__restrict __format, __gnuc_va_list __arg) __asm__ ("" "__isoc99_vscanf")

     __attribute__ ((__format__ (__scanf__, 1, 0))) ;
extern int vsscanf (const char *__restrict __s, const char *__restrict __format, __gnuc_va_list __arg) __asm__ ("" "__isoc99_vsscanf") __attribute__ ((__nothrow__ , __leaf__))



     __attribute__ ((__format__ (__scanf__, 2, 0)));
# 477 "/usr/include/stdio.h" 3 4
extern int fgetc (FILE *__stream);
extern int getc (FILE *__stream);





extern int getchar (void);



#define getc(_fp) _IO_getc (_fp)
# 517 "/usr/include/stdio.h" 3 4
extern int fputc (int __c, FILE *__stream);
extern int putc (int __c, FILE *__stream);





extern int putchar (int __c);



#define putc(_ch,_fp) _IO_putc (_ch, _fp)
# 564 "/usr/include/stdio.h" 3 4
extern char *fgets (char *__restrict __s, int __n, FILE *__restrict __stream)
     ;
# 577 "/usr/include/stdio.h" 3 4
extern char *gets (char *__s) __attribute__ ((__deprecated__));
# 626 "/usr/include/stdio.h" 3 4
extern int fputs (const char *__restrict __s, FILE *__restrict __stream);





extern int puts (const char *__s);






extern int ungetc (int __c, FILE *__stream);






extern size_t fread (void *__restrict __ptr, size_t __size,
       size_t __n, FILE *__restrict __stream) ;




extern size_t fwrite (const void *__restrict __ptr, size_t __size,
        size_t __n, FILE *__restrict __s);
# 684 "/usr/include/stdio.h" 3 4
extern int fseek (FILE *__stream, long int __off, int __whence);




extern long int ftell (FILE *__stream) ;




extern void rewind (FILE *__stream);
# 731 "/usr/include/stdio.h" 3 4
extern int fgetpos (FILE *__restrict __stream, fpos_t *__restrict __pos);




extern int fsetpos (FILE *__stream, const fpos_t *__pos);
# 757 "/usr/include/stdio.h" 3 4
extern void clearerr (FILE *__stream) __attribute__ ((__nothrow__ , __leaf__));

extern int feof (FILE *__stream) __attribute__ ((__nothrow__ , __leaf__)) ;

extern int ferror (FILE *__stream) __attribute__ ((__nothrow__ , __leaf__)) ;
# 775 "/usr/include/stdio.h" 3 4
extern void perror (const char *__s);





# 1 "/usr/include/x86_64-linux-gnu/bits/sys_errlist.h" 1 3 4
# 782 "/usr/include/stdio.h" 2 3 4
# 868 "/usr/include/stdio.h" 3 4

# 4 "checkinput.c" 2
# 1 "/usr/include/stdlib.h" 1 3 4
# 24 "/usr/include/stdlib.h" 3 4
#define __GLIBC_INTERNAL_STARTING_HEADER_IMPLEMENTATION 
# 1 "/usr/include/x86_64-linux-gnu/bits/libc-header-start.h" 1 3 4
# 31 "/usr/include/x86_64-linux-gnu/bits/libc-header-start.h" 3 4
#undef __GLIBC_INTERNAL_STARTING_HEADER_IMPLEMENTATION





#undef __GLIBC_USE_LIB_EXT2




#define __GLIBC_USE_LIB_EXT2 0




#undef __GLIBC_USE_IEC_60559_BFP_EXT



#define __GLIBC_USE_IEC_60559_BFP_EXT 0




#undef __GLIBC_USE_IEC_60559_FUNCS_EXT



#define __GLIBC_USE_IEC_60559_FUNCS_EXT 0




#undef __GLIBC_USE_IEC_60559_TYPES_EXT



#define __GLIBC_USE_IEC_60559_TYPES_EXT 0
# 26 "/usr/include/stdlib.h" 2 3 4


#define __need_size_t 
#define __need_wchar_t 
#define __need_NULL 
# 1 "/usr/lib/gcc/x86_64-linux-gnu/7/include/stddef.h" 1 3 4
# 238 "/usr/lib/gcc/x86_64-linux-gnu/7/include/stddef.h" 3 4
#undef __need_size_t
# 267 "/usr/lib/gcc/x86_64-linux-gnu/7/include/stddef.h" 3 4
#define __wchar_t__ 
#define __WCHAR_T__ 
#define _WCHAR_T 
#define _T_WCHAR_ 
#define _T_WCHAR 
#define __WCHAR_T 
#define _WCHAR_T_ 
#define _BSD_WCHAR_T_ 
#define _WCHAR_T_DEFINED_ 
#define _WCHAR_T_DEFINED 
#define _WCHAR_T_H 
#define ___int_wchar_t_h 
#define __INT_WCHAR_T_H 
#define _GCC_WCHAR_T 
#define _WCHAR_T_DECLARED 
# 294 "/usr/lib/gcc/x86_64-linux-gnu/7/include/stddef.h" 3 4
#undef _BSD_WCHAR_T_
# 328 "/usr/lib/gcc/x86_64-linux-gnu/7/include/stddef.h" 3 4
typedef int wchar_t;
# 347 "/usr/lib/gcc/x86_64-linux-gnu/7/include/stddef.h" 3 4
#undef __need_wchar_t
# 401 "/usr/lib/gcc/x86_64-linux-gnu/7/include/stddef.h" 3 4
#undef NULL




#define NULL ((void *)0)





#undef __need_NULL
# 32 "/usr/include/stdlib.h" 2 3 4



#define _STDLIB_H 1
# 55 "/usr/include/stdlib.h" 3 4
# 1 "/usr/include/x86_64-linux-gnu/bits/floatn.h" 1 3 4
# 20 "/usr/include/x86_64-linux-gnu/bits/floatn.h" 3 4
#define _BITS_FLOATN_H 
# 33 "/usr/include/x86_64-linux-gnu/bits/floatn.h" 3 4
#define __HAVE_FLOAT128 1







#define __HAVE_DISTINCT_FLOAT128 1







#define __HAVE_FLOAT64X 1





#define __HAVE_FLOAT64X_LONG_DOUBLE 1
# 66 "/usr/include/x86_64-linux-gnu/bits/floatn.h" 3 4
#define __f128(x) x ##f128
# 78 "/usr/include/x86_64-linux-gnu/bits/floatn.h" 3 4
#define __CFLOAT128 _Complex _Float128
# 120 "/usr/include/x86_64-linux-gnu/bits/floatn.h" 3 4
# 1 "/usr/include/x86_64-linux-gnu/bits/floatn-common.h" 1 3 4
# 21 "/usr/include/x86_64-linux-gnu/bits/floatn-common.h" 3 4
#define _BITS_FLOATN_COMMON_H 


# 1 "/usr/include/x86_64-linux-gnu/bits/long-double.h" 1 3 4
# 25 "/usr/include/x86_64-linux-gnu/bits/floatn-common.h" 2 3 4
# 34 "/usr/include/x86_64-linux-gnu/bits/floatn-common.h" 3 4
#define __HAVE_FLOAT16 0
#define __HAVE_FLOAT32 1
#define __HAVE_FLOAT64 1
#define __HAVE_FLOAT32X 1
#define __HAVE_FLOAT128X 0
# 52 "/usr/include/x86_64-linux-gnu/bits/floatn-common.h" 3 4
#define __HAVE_DISTINCT_FLOAT16 __HAVE_FLOAT16
#define __HAVE_DISTINCT_FLOAT32 0
#define __HAVE_DISTINCT_FLOAT64 0
#define __HAVE_DISTINCT_FLOAT32X 0
#define __HAVE_DISTINCT_FLOAT64X 0
#define __HAVE_DISTINCT_FLOAT128X __HAVE_FLOAT128X





#define __HAVE_FLOATN_NOT_TYPEDEF 1
# 86 "/usr/include/x86_64-linux-gnu/bits/floatn-common.h" 3 4
#define __f32(x) x ##f32
# 98 "/usr/include/x86_64-linux-gnu/bits/floatn-common.h" 3 4
#define __f64(x) x ##f64







#define __f32x(x) x ##f32x
# 118 "/usr/include/x86_64-linux-gnu/bits/floatn-common.h" 3 4
#define __f64x(x) x ##f64x
# 144 "/usr/include/x86_64-linux-gnu/bits/floatn-common.h" 3 4
#define __CFLOAT32 _Complex _Float32
# 156 "/usr/include/x86_64-linux-gnu/bits/floatn-common.h" 3 4
#define __CFLOAT64 _Complex _Float64







#define __CFLOAT32X _Complex _Float32x
# 176 "/usr/include/x86_64-linux-gnu/bits/floatn-common.h" 3 4
#define __CFLOAT64X _Complex _Float64x
# 121 "/usr/include/x86_64-linux-gnu/bits/floatn.h" 2 3 4
# 56 "/usr/include/stdlib.h" 2 3 4


typedef struct
  {
    int quot;
    int rem;
  } div_t;



typedef struct
  {
    long int quot;
    long int rem;
  } ldiv_t;
#define __ldiv_t_defined 1




__extension__ typedef struct
  {
    long long int quot;
    long long int rem;
  } lldiv_t;
#define __lldiv_t_defined 1




#define RAND_MAX 2147483647




#define EXIT_FAILURE 1
#define EXIT_SUCCESS 0



#define MB_CUR_MAX (__ctype_get_mb_cur_max ())
extern size_t __ctype_get_mb_cur_max (void) __attribute__ ((__nothrow__ , __leaf__)) ;



extern double atof (const char *__nptr)
     __attribute__ ((__nothrow__ , __leaf__)) __attribute__ ((__pure__)) __attribute__ ((__nonnull__ (1))) ;

extern int atoi (const char *__nptr)
     __attribute__ ((__nothrow__ , __leaf__)) __attribute__ ((__pure__)) __attribute__ ((__nonnull__ (1))) ;

extern long int atol (const char *__nptr)
     __attribute__ ((__nothrow__ , __leaf__)) __attribute__ ((__pure__)) __attribute__ ((__nonnull__ (1))) ;



__extension__ extern long long int atoll (const char *__nptr)
     __attribute__ ((__nothrow__ , __leaf__)) __attribute__ ((__pure__)) __attribute__ ((__nonnull__ (1))) ;



extern double strtod (const char *__restrict __nptr,
        char **__restrict __endptr)
     __attribute__ ((__nothrow__ , __leaf__)) __attribute__ ((__nonnull__ (1)));



extern float strtof (const char *__restrict __nptr,
       char **__restrict __endptr) __attribute__ ((__nothrow__ , __leaf__)) __attribute__ ((__nonnull__ (1)));

extern long double strtold (const char *__restrict __nptr,
       char **__restrict __endptr)
     __attribute__ ((__nothrow__ , __leaf__)) __attribute__ ((__nonnull__ (1)));
# 176 "/usr/include/stdlib.h" 3 4
extern long int strtol (const char *__restrict __nptr,
   char **__restrict __endptr, int __base)
     __attribute__ ((__nothrow__ , __leaf__)) __attribute__ ((__nonnull__ (1)));

extern unsigned long int strtoul (const char *__restrict __nptr,
      char **__restrict __endptr, int __base)
     __attribute__ ((__nothrow__ , __leaf__)) __attribute__ ((__nonnull__ (1)));
# 199 "/usr/include/stdlib.h" 3 4
__extension__
extern long long int strtoll (const char *__restrict __nptr,
         char **__restrict __endptr, int __base)
     __attribute__ ((__nothrow__ , __leaf__)) __attribute__ ((__nonnull__ (1)));

__extension__
extern unsigned long long int strtoull (const char *__restrict __nptr,
     char **__restrict __endptr, int __base)
     __attribute__ ((__nothrow__ , __leaf__)) __attribute__ ((__nonnull__ (1)));
# 453 "/usr/include/stdlib.h" 3 4
extern int rand (void) __attribute__ ((__nothrow__ , __leaf__));

extern void srand (unsigned int __seed) __attribute__ ((__nothrow__ , __leaf__));
# 539 "/usr/include/stdlib.h" 3 4
extern void *malloc (size_t __size) __attribute__ ((__nothrow__ , __leaf__)) __attribute__ ((__malloc__)) ;

extern void *calloc (size_t __nmemb, size_t __size)
     __attribute__ ((__nothrow__ , __leaf__)) __attribute__ ((__malloc__)) ;






extern void *realloc (void *__ptr, size_t __size)
     __attribute__ ((__nothrow__ , __leaf__)) __attribute__ ((__warn_unused_result__));
# 563 "/usr/include/stdlib.h" 3 4
extern void free (void *__ptr) __attribute__ ((__nothrow__ , __leaf__));
# 588 "/usr/include/stdlib.h" 3 4
extern void abort (void) __attribute__ ((__nothrow__ , __leaf__)) __attribute__ ((__noreturn__));



extern int atexit (void (*__func) (void)) __attribute__ ((__nothrow__ , __leaf__)) __attribute__ ((__nonnull__ (1)));
# 614 "/usr/include/stdlib.h" 3 4
extern void exit (int __status) __attribute__ ((__nothrow__ , __leaf__)) __attribute__ ((__noreturn__));
# 626 "/usr/include/stdlib.h" 3 4
extern void _Exit (int __status) __attribute__ ((__nothrow__ , __leaf__)) __attribute__ ((__noreturn__));




extern char *getenv (const char *__name) __attribute__ ((__nothrow__ , __leaf__)) __attribute__ ((__nonnull__ (1))) ;
# 781 "/usr/include/stdlib.h" 3 4
extern int system (const char *__command) ;
# 804 "/usr/include/stdlib.h" 3 4
#define __COMPAR_FN_T 
typedef int (*__compar_fn_t) (const void *, const void *);
# 817 "/usr/include/stdlib.h" 3 4
extern void *bsearch (const void *__key, const void *__base,
        size_t __nmemb, size_t __size, __compar_fn_t __compar)
     __attribute__ ((__nonnull__ (1, 2, 5))) ;







extern void qsort (void *__base, size_t __nmemb, size_t __size,
     __compar_fn_t __compar) __attribute__ ((__nonnull__ (1, 4)));
# 837 "/usr/include/stdlib.h" 3 4
extern int abs (int __x) __attribute__ ((__nothrow__ , __leaf__)) __attribute__ ((__const__)) ;
extern long int labs (long int __x) __attribute__ ((__nothrow__ , __leaf__)) __attribute__ ((__const__)) ;


__extension__ extern long long int llabs (long long int __x)
     __attribute__ ((__nothrow__ , __leaf__)) __attribute__ ((__const__)) ;






extern div_t div (int __numer, int __denom)
     __attribute__ ((__nothrow__ , __leaf__)) __attribute__ ((__const__)) ;
extern ldiv_t ldiv (long int __numer, long int __denom)
     __attribute__ ((__nothrow__ , __leaf__)) __attribute__ ((__const__)) ;


__extension__ extern lldiv_t lldiv (long long int __numer,
        long long int __denom)
     __attribute__ ((__nothrow__ , __leaf__)) __attribute__ ((__const__)) ;
# 919 "/usr/include/stdlib.h" 3 4
extern int mblen (const char *__s, size_t __n) __attribute__ ((__nothrow__ , __leaf__));


extern int mbtowc (wchar_t *__restrict __pwc,
     const char *__restrict __s, size_t __n) __attribute__ ((__nothrow__ , __leaf__));


extern int wctomb (char *__s, wchar_t __wchar) __attribute__ ((__nothrow__ , __leaf__));



extern size_t mbstowcs (wchar_t *__restrict __pwcs,
   const char *__restrict __s, size_t __n) __attribute__ ((__nothrow__ , __leaf__));

extern size_t wcstombs (char *__restrict __s,
   const wchar_t *__restrict __pwcs, size_t __n)
     __attribute__ ((__nothrow__ , __leaf__));
# 1016 "/usr/include/stdlib.h" 3 4
# 1 "/usr/include/x86_64-linux-gnu/bits/stdlib-float.h" 1 3 4
# 1017 "/usr/include/stdlib.h" 2 3 4
# 1026 "/usr/include/stdlib.h" 3 4

# 5 "checkinput.c" 2





# 9 "checkinput.c"
void checkInput(int err) {
  if (!err || err == 
# 10 "checkinput.c" 3 4
                    (-1)
# 10 "checkinput.c"
                       ) {
    printf("\nInvalid input!\n");
    exit(1);
  }
}
\end{verbatim}
\end{document}
